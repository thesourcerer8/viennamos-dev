\chapter{Automated OpenCL User-Kernel Generation} \label{chap:kernel-generation}

While {\ViennaCL} provides a convenient means of including custom {\OpenCL} compute kernels, cf.~Chap.~\ref{chap:custom},
it can be rather tedious to come up with a good compute kernel, or to come up with many similar kernels differing in small details only.
For the case of BLAS level 1 and level 2 operations, {\ViennaCL} now provides an automated kernel generator, which takes a high-level specification of the operations and create one or more suitable OpenCL kernels.
This allows for high-performance implementations of algorithms which may otherwise lead to spurious temporary objects.

Consider the operation
\begin{align*}
\mathbf{x} = \mathbf{A} \times \bigl[ (\mathbf{y} \cdot (\mathbf{y}+\mathbf{z}))\mathbf{y} + \mathbf{z} \bigr] \ ,
\end{align*}
where $\mathbf{x}$, $\mathbf{y}$ and $\mathbf{z}$ denote vectors, $\mathbf{A}$ is a dense matrix, and the dot denotes the vector dot product.
With the generator it is sufficient to write the following C++ code in order to obtain an OpenCL kernel:
\begin{lstlisting}
// Instantiation of the symbolic variables
symbolic_vector<NumericT, 0> sX;
symbolic_matrix<NumericT, 1> sA;
symbolic_vector<NumericT, 2> sY;
symbolic_vector<NumericT, 3> sZ;

//Creation of the custom operation
custom_operation my_op( sX = prod(sA, inner_prod(sY, sY+sZ) * sY + sZ),
                        "operation_name" );
\end{lstlisting}
where \lstinline|NumericT| is either \lstinline|float| or \lstinline|double|.
The string provided as second parameter is required and can be used to identify, manage and retrieve different kernels.
No two \lstinline|custom_operation|s are allowed to be identified using the same string.

The custom operation object \lstinline|my_op| can be enqueued like any other kernel:
\begin{lstlisting}
//Execution of the custom operation
viennacl::ocl::enqueue(my_op(x,A,y,z));
\end{lstlisting}
Here, \lstinline|x|, \lstinline|y|, \lstinline|z| are of type \lstinline|viennacl::vector<NumericT>| and \lstinline|A| is of type \lstinline|viennacl::matrix<NumericT>|.

\TIP{Sample code can be found in \lstinline|tests/src/generator_*.cpp|}

\NOTE{ The kernel generator is still experimental, yet already able to generate rather complex compute kernels. }

