
\chapter{Versioning}

Each release of {\ViennaGrid} carries a three-fold version number, given by\\
\begin{center}
 \texttt{ViennaGrid X.Y.Z} . \\
\end{center}
For users migrating from an older release of {\ViennaGrid} to a newer one, the
following guidelines apply:
\begin{itemize}
 \item \texttt{X} is the \emph{major version number}, starting with \texttt{1}.
A change in the major version number is not necessarily API-compatible with any
versions of {\ViennaGrid} carrying a different major version number. In particular,
end users of {\ViennaGrid} have to expect considerable code changes when changing
between different major versions of {\ViennaGrid}.

 \item \texttt{Y} denotes the \emph{minor version number}, restarting with zero
whenever the major version number changes. The minor version number is
incremented whenever significant functionality is added to {\ViennaGrid}.
The API of an older release of {\ViennaGrid} with smaller minor version number
(but same major version number) is \emph{essentially} compatible to the new
version, hence end users of {\ViennaGrid} usually do not have to alter their
application code. There may be small adjustments in the public API,
which will be extensively documented in the change logs and require
at most very little changes in the application code.

 \item \texttt{Z} is the \emph{revision number}. If either the major or the
minor version number changes, the revision number is reset to zero. The public APIs of releases of
{\ViennaGrid}, which only differ in their revision number, are compatible.
Typically, the revision number is increased whenever bugfixes are applied,
performance and/or memory footprint is improved, or some extra, not overly significant functionality is
added.
\end{itemize}

\TIP{Always try to use the latest version of {\ViennaGrid} before submitting bug
reports!}
