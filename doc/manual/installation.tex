\chapter{Installation}

This chapter shows how {\ViennaCL} can be integrated into a project and how the
examples are built. The necessary steps are outlined for several different
platforms, but we could not check every possible combination of hardware,
operating system and compiler. If you experience any trouble, please write to
the maining list at \\
\begin{center}
\texttt{viennacl-support$@$lists.sourceforge.net}
\end{center}


% -----------------------------------------------------------------------------
% -----------------------------------------------------------------------------
\section{Dependencies}
% -----------------------------------------------------------------------------
% -----------------------------------------------------------------------------
\label{dependencies}
{\ViennaCL} uses the {\CMake} build system for multi-platform support.
Thus, before you proceed with the installation of {\ViennaCL}, make sure you
have a recent version of {\CMake} installed.

To use {\ViennaCL}, only the following minimal prerequisite has to be fulfilled:
\begin{itemize}
 \item A fairly recent C++ compiler (e.g.~{\GCC} version 4.2.x or above and Visual C++
2005 and 2010 are known to work)
\end{itemize}


The full potential of {\ViennaCL} is available with the following optional libraries:
\begin{itemize}
 \item {\CMake}~\cite{cmake} as build system (optional, but highly recommended for building examples)
 \item {\OpenCL}~\cite{khronoscl,nvidiacl} for accessing compute devices (GPUs); see Section~\ref{opencllibs} for details.
 \item {\CUDA}~\cite{nvidiacuda} for using CUDA-accelerated operations.
 \item {\OpenMP}~\cite{openmp} for directive-based parallelism on CPUs.
 \item {\ublas} (shipped with {\Boost}~\cite{boost}) provides the same interface as {\ViennaCL} and allows to switch between CPU and GPU seamlessly, see the tutorials.
 \item Eigen \cite{eigen} can be used to fill {\ViennaCL} types directly. Moreover, the iterative solvers in {\ViennaCL} can directly be used with Eigen objects.
 \item MTL 4 \cite{mtl4} can be used to fill {\ViennaCL} types directly. Even though MTL 4 provides its own iterative solvers, the {\ViennaCL} solvers can also be used with MTL 4 objects.
\end{itemize}

%The use of {\OpenMP} for the benchmark suite allows fair comparisons between your multi-core CPU and your compute device (e.g.~GPU).

\section{Generic Installation of ViennaCL} \label{sec:viennacl-installation}
Since {\ViennaCL} is essentially a header-only library (the only exception is
described in Chapter \ref{chap:tuning}), it is sufficient to copy the folder
\lstinline|viennacl/| either into your project folder or to your global system
include path. On Unix based systems, this is often \lstinline|/usr/include/| or
\lstinline|/usr/local/include/|. If the OpenCL headers are not installed on your system,
you should repeat the above procedure with the folder \lstinline|CL/|.

On Windows, the situation strongly depends on your development environment. We advise users
to consult the documentation of their compiler on how to set the include path correctly.
With Visual Studio this is usually something like \texttt{C:$\setminus$Program Files$\setminus$Microsoft Visual Studio 9.0$\setminus$VC$\setminus$include}
and can be set in \texttt{Tools -> Options -> Projects and Solutions -> VC++-\-Directories}.
For using the {\CUDA} backend, simply make sure that the {\CUDA} SDK is installed properly.
If you wish to use the {\OpenCL} backend, the include and library directories of your {\OpenCL} SDK should also be added there.

\NOTE{If multiple {\OpenCL} libraries are available on the host system, {\ViennaCL} uses the first platform returned by the system. Consult Chap.~\ref{chap:multi-devices} for configuring the use of other platforms.}


% -----------------------------------------------------------------------------
% -----------------------------------------------------------------------------
\section{Get the {\OpenCL} Library}
\label{opencllibs}
% -----------------------------------------------------------------------------
% -----------------------------------------------------------------------------
In order to compile and run {\OpenCL} applications, a corresponding library
(e.g.~\texttt{libOpenCL.so} under Unix based systems) and is required.
If {\OpenCL} is to be used with GPUs, suitable drivers have to be installed. This section describes how these can be acquired.

\TIP{Note, that for Mac OS X systems there is no need to install an {\OpenCL}
capable driver and the corresponding library.
The {\OpenCL} library is already present if a suitable graphics
card is present. The setup of {\ViennaCL} on Mac OS X is discussed in
Section~\ref{apple}.}

\subsection{\NVIDIA Driver}
\NVIDIA provides the {\OpenCL} library with the GPU driver. Therefore, if a
\NVIDIA driver is present on the system, the library is too. However,
not all of the released drivers contain the {\OpenCL} library.
A driver which is known to support {\OpenCL}, and hence providing the required
library, is $260.19.21$. Note that the latest {\NVIDIA} drivers do not include
the {\OpenCL} headers anymore. Therefore, the official {\OpenCL} headers from
the Khronos group \cite{khronoscl} are also shipped with {\ViennaCL} in the
folder \lstinline|CL/|.

\subsection{AMD Accelerated Parallel Processing SDK (formerly Stream SDK)} \label{sec:opencl-on-ati}
AMD has provided the {\OpenCL} library with the Accelerated Parallel Processing (APP)
SDK~\cite{atistream} previously, now the {\OpenCL} library is also included in the GPU driver.
At the release of {\ViennaCLversion}, the latest version of the
SDK is $2.7$. If used with AMD GPUs, recent AMD GPU drivers are typically required. If {\ViennaCL} is to be run on multi-core CPUs,
no additional GPU driver is required. The installation notes
of the APP SDK provides guidance throughout the
installation process~\cite{atistreamdocu}.

\TIP{If the SDK is installed in a non-system wide location on UNIX-based systems, be sure to add the
{\OpenCL} library path to the \texttt{LD\_LIBRARY\_PATH} environment variable.
Otherwise, linker errors will occur as the required library cannot be found.}

It is important to note that the AMD APP SDK may not provide {\OpenCL}
certified double precision support~\cite{atidouble} on some CPUs and GPUs.

\NOTE{Unfortunately, some versions of the AMD APP SDK are known to have bugs. For example, APP SDK 2.7 on Linux causes BiCGStab to fail on some devices.}

\subsection{INTEL OpenCL SDK} \label{sec:opencl-on-intel}
 {\ViennaCL} works fine with the INTEL OpenCL SDK on Windows and Linux.
The correct linker path is set automatically in \lstinline|CMakeLists.txt| when using the {\CMake} build system, cf.~Sec.~\ref{sec:viennacl-installation}.

% -----------------------------------------------------------------------------
% -----------------------------------------------------------------------------
\section{Enabling OpenMP, OpenCL, or CUDA Backends} \label{sec:cuda-opencl-backends}
% -----------------------------------------------------------------------------
% -----------------------------------------------------------------------------

\TIP{The new default behavior in {\ViennaCL} 1.4.0 is to use the CPU backend. {\OpenCL} and {\CUDA} backends need to be enabled by appropriate preprocessor \lstinline|#define|s.}

By default, {\ViennaCL} now uses the single-threaded/OpenMP-enabled CPU backend.
The {\OpenCL} and the {\CUDA}-backend need to be enabled explicitly by using preprocessor constants as follows:

\begin{center}
\begin{tabular}{|l|l|}
 \hline
   \textbf{Preprocessor} \lstinline|#define| & \textbf{Default computing backend} \\
 \hline
 none                              & CPU, single-threaded \\
 \hline
 \lstinline|VIENNACL_WITH_OPENMP|  & CPU with OpenMP (compiler flags required) \\
 \hline
 \lstinline|VIENNACL_WITH_OPENCL|  & OpenCL \\
 \hline
 \lstinline|VIENNACL_WITH_CUDA|  & CUDA \\
 \hline
\end{tabular}
\end{center}

The preprocessor constants can be either defined at the beginning of the source file (prior to any ViennaCL-includes), or passed to the compiler as command line argument.
For example, on \lstinline|g++| the respective command line option for enabling the OpenCL backend is \lstinline|-DVIENNACL_WITH_OPENCL|.
Note that CUDA requires the \lstinline|nvcc| compiler. Furthermore, the use of {\OpenMP} usually requires additional compiler flags (on \lstinline|g++| this is for example \lstinline|-fopenmp|).

\TIP{The CUDA backend requires a compilation using \lstinline|nvcc|.}

Multiple backends can be used simultaneously. In such case, \lstinline|CUDA| has higher priority than \lstinline|OpenCL|, which has higher priority over the CPU backend when it comes to selecting the default backend.


% -----------------------------------------------------------------------------
% -----------------------------------------------------------------------------
\section{Building the Examples and Tutorials}
% -----------------------------------------------------------------------------
% -----------------------------------------------------------------------------
For building the examples, we suppose that {\CMake} is properly set up
on your system. The other dependencies are listed in Tab.~\ref{tab:tutorial-dependencies}.

\begin{table}[tb]
\begin{center}
\begin{tabular}{l|l}
Example/Tutorial & Dependencies\\
\hline
\texttt{tutorial/amg.cpp}        & {\OpenCL}, {\ublas} \\
\texttt{tutorial/bandwidth-reduction.cpp} & - \\
\texttt{tutorial/blas1.cpp/cu}      & - \\
\texttt{tutorial/blas2.cpp/cu}      & {\ublas} \\
\texttt{tutorial/blas3.cpp/cu}      & {\ublas} \\
\texttt{tutorial/custom-kernels.cpp}       & {\OpenCL} \\
\texttt{tutorial/custom-context.cpp}       & {\OpenCL} \\
\texttt{tutorial/eigen-with-viennacl.cpp}  & {\Eigen} \\
\texttt{tutorial/fft.cpp}                  & {\OpenCL} \\
\texttt{tutorial/iterative.cpp/cu}         & {\ublas} \\
\texttt{tutorial/iterative-ublas.cpp}      & {\ublas}  \\
\texttt{tutorial/iterative-eigen.cpp}      & {\Eigen}   \\
\texttt{tutorial/iterative-mtl4.cpp}       & {\MTL}    \\
\texttt{tutorial/lanczos.cpp/cu}           & {\ublas}    \\
\texttt{tutorial/least-squares.cpp/cu}     & {\ublas}    \\
\texttt{tutorial/matrix-range.cpp/cu}      & {\ublas} \\
\texttt{tutorial/mtl4-with-viennacl.cpp}   & {\MTL} \\
\texttt{tutorial/power-iter.cpp/cu}        & {\ublas} \\
\texttt{tutorial/qr.cpp/cu}                & {\ublas} \\
\texttt{tutorial/spai.cpp}                 & {\OpenCL}, {\ublas} \\
\texttt{tutorial/vector-range.cpp/cu}      & {\ublas} \\
\texttt{tutorial/viennacl-info.cpp}        & {\OpenCL} \\
\texttt{benchmarks/blas3.cpp/cu}   & - \\
\texttt{benchmarks/opencl.cpp}     & {\OpenCL} \\
\texttt{benchmarks/solver.cpp/cu}  & {\ublas} \\
\texttt{benchmarks/sparse.cpp/cu}  & {\ublas} \\
\texttt{benchmarks/vector.cpp/cu}  & - \\
\end{tabular}
\caption{Dependencies for the examples in the \texttt{examples/} folder. Examples using the CUDA-backend use the \lstinline|.cu| file extension.
Note that all examples can be run using either of the CPU, OpenCL, and CUDA backend unless an explicit {\OpenCL}-dependency is stated.}
\label{tab:tutorial-dependencies}
\end{center}
\end{table}

Before building the examples, customize \texttt{CMakeLists.txt} in the {\ViennaCL} root folder for your needs.
Per default, all examples using {\ublas}, {Eigen} and {MTL4} are turned off.
Please enable the respective examples based on the libraries available on your machine.
Directions on how to accomplish this are given directly within the \texttt{CMakeLists.txt} file.
A brief overview of the most important flags is as follows:

\begin{center}
 \begin{tabular}{|l|l|}
  \hline
  {\CMake} Flag & Purpose \\
  \hline
  \lstinline|ENABLE_CUDA|   & Builds examples with the {\CUDA} backend enabled\\
  \lstinline|ENABLE_OPENCL| & Builds examples with the {\OpenCL} backend enabled\\
  \lstinline|ENABLE_OPENMP| & Builds examples with {\OpenMP} for the CPU backend enabled\\
  \hline
  \lstinline|ENABLE_EIGEN|  & Builds examples depending on {\Eigen}\\
  \lstinline|ENABLE_MTL4|   & Builds examples depending on {\MTL}\\
  \lstinline|ENABLE_UBLAS|  & Builds examples depending on {\ublas}\\
  \hline
 \end{tabular}
\end{center}


\subsection{Linux}
To build the examples, open a terminal and change to:
\begin{lstlisting}
 $> cd /your-ViennaCL-path/build/
\end{lstlisting}
Execute
\begin{lstlisting}
 $> cmake ..
\end{lstlisting}
to obtain a Makefile and type
\begin{lstlisting}
 $> make
\end{lstlisting}
to build the examples. If some of the dependencies in Tab.~\ref{tab:tutorial-dependencies} are not fulfilled, you can build each example separately:
\begin{lstlisting}
 $> make blas1             #builds the blas level 1 tutorial
 $> make vectorbench       #builds vector benchmarks
\end{lstlisting}

\TIP{Speed up the building process by using jobs, e.g. \keyword{make -j4}.}

Execute the examples from the \lstinline|build/| folder as follows:
\begin{lstlisting}
 $> examples/tutorial/blas1
 $> examples/benchmarks/vectorbench
\end{lstlisting}
Note that all benchmark executables carry the suffix \lstinline|bench|.

\TIP{Use the {\CMake}-GUI via \lstinline|cmake-gui ..| within the \lstinline|build/| folder in order to enable or disable optional libraries conveniently.}

\subsection{Mac OS X}
\label{apple}
The tools mentioned in Section \ref{dependencies} are available on
Macintosh platforms too.
For the {\GCC} compiler the Xcode~\cite{xcode} package has to be installed.
To install {\CMake} and {\Boost} external portation tools have to be used,
for example, Fink~\cite{fink}, DarwinPorts~\cite{darwinports}
or MacPorts~\cite{macports}. Such portation tools provide the
aforementioned packages, {\CMake} and {\Boost}, for macintosh platforms.

\TIP{If the {\CMake} build system has problems detecting your {\Boost} libraries,
determine the location of your {\Boost} folder.
Open the \texttt{CMakeLists.txt} file in the root directory of {\ViennaCL} and
add your {\Boost} path after the following entry:
\texttt{IF(\${CMAKE\_SYSTEM\_NAME} MATCHES "Darwin")} }

The build process of {\ViennaCL} on Mac OS is similar to Linux.

\subsection{Windows}
In the following the procedure is outlined for \texttt{Visual Studio}: Assuming that an {\OpenCL} SDK and {\CMake} is already installed, Visual Studio solution and project files can be created using {\CMake}:
\begin{itemize}
\item Open the {\CMake} GUI.
\item Set the {\ViennaCL} base directory as source directory.
\item Set the \texttt{build/} directory as build directory.
\item Click on 'Configure' and select the appropriate generator (e.g.~\texttt{Visual Studio 9 2008}).
\item If you set \lstinline|ENABLE_CUDA|, \lstinline|ENABLE_CUDA|, \lstinline|ENABLE_MTL4|, or \lstinline|ENABLE_OPENCL| and the paths cannot be found, please select the advanced view and provide the required paths manually.
\item If you set \lstinline|ENABLE_UBLAS| and the paths cannot be found, please select the advanced view and provide the required paths manually. You may have to specify the linker path for Boost manually within your Visual Studio IDE.
\item Click again on 'Configure'. You should not receive an error at this point.
\item Click on 'Generate'.
\item The project files can now be found in the {\ViennaCL} build directory, where they can be opened and compiled with Visual Studio (provided that the include and library paths are set correctly, see Sec.~\ref{sec:viennacl-installation}).
\end{itemize}

\TIP{The examples and tutorials should be executed from within the \texttt{build/} directory of {\ViennaCL}, otherwise the sample data files cannot be found.}






