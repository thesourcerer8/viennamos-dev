\chapter{Installation}

This chapter shows how {\ViennaMath} can be integrated into a project and how
the examples are built. The necessary steps are outlined for several different
platforms, but we could not check every possible combination of hardware,
operating system, and compiler. If you experience any trouble, please write to
the mailing list at \\
\begin{center}
\texttt{viennamath-support$@$lists.sourceforge.net} 
\end{center}


% -----------------------------------------------------------------------------
% -----------------------------------------------------------------------------
\section{Dependencies}
% -----------------------------------------------------------------------------
% -----------------------------------------------------------------------------
\label{dependencies}

\begin{itemize}
 \item A recent C++ compiler (e.g.~{\GCC} version 4.2.x or above and Visual C++
2008 are known to work)
 \item {\CMake}~\cite{cmake} as build system (optional, but recommended
for building the examples)
\end{itemize}


\section{Generic Installation of ViennaMath} \label{sec:viennamath-installation}
Since {\ViennaMath} is a header-only library, it is sufficient to copy the 
\lstinline|viennamath/| folder either into your project folder or to your global system
include path. On Unix based systems, this is often \lstinline|/usr/include/| or
\lstinline|/usr/local/include/|.

On Windows, the situation strongly depends on your development environment.
Please consult the documentation of your compiler or development environment on how to set the include
path correctly. The include paths in Visual Studio are usually something like
\texttt{C:$\setminus$Program Files$\setminus$Microsoft Visual Studio
9.0$\setminus$VC$\setminus$include}
and can be set in \texttt{Tools -> Options -> Projects and Solutions ->
VC++-\-Directories}. 


% -----------------------------------------------------------------------------
% -----------------------------------------------------------------------------
\section{Building the Examples and Tutorials}
% -----------------------------------------------------------------------------
% -----------------------------------------------------------------------------
An overview of available examples and their purpose is given in
Tab.~\ref{tab:tutorial-dependencies}.
For building the examples, we suppose that {\CMake} is properly set up
on your system. In the following, instructions on how to build the examples on different platforms are given.

\begin{table}[tb]
\begin{center}
\begin{tabular}{l|p{8.5cm}}
File & Purpose\\
\hline
\texttt{basic.cpp}              & Basic handling of {\ViennaMath} expressions \\
\texttt{latex\_output.cpp}      & How to use and customize the \LaTeX translator \\
\texttt{model\_benchmark.cpp}    & An example of how {\ViennaMath} can eliminate dependencies in an expression \\
\texttt{newton\_solve.cpp}      & A Newton solver using {\ViennaMath} expressions \\
\texttt{traversal.cpp}          & How to traverse a {\ViennaMath} expressions \\
\texttt{substitute.cpp}         & Substitute terms in a {\ViennaMath} expressions \\
\texttt{vector\_expr.cpp}       & Explains the use of vector expressions \\
\end{tabular}
\caption{Overview of the examples in the \texttt{examples/} folder}
\label{tab:tutorial-dependencies}
\end{center}
\end{table}

\subsection{Linux}
To build the examples, open a terminal and change to:

\begin{lstlisting}
 $> cd /your-ViennaMath-path/build/
\end{lstlisting}
Execute
\begin{lstlisting}
 $> cmake ..
\end{lstlisting}
to obtain a Makefile and type
\begin{lstlisting}
 $> make 
\end{lstlisting}
to build the examples. If desired, one can build each example separately instead:
\begin{lstlisting}
 $> make basic            #builds the 'basic' tutorial
 $> make substitute       #builds the 'substitute' tutorial
\end{lstlisting}

\TIP{Speed up the building process by using jobs, e.g. \keyword{make -j4}.}

\subsection{Mac OS X}
\label{apple}
The tools mentioned in Section \ref{dependencies} are available on 
Macintosh platforms too. 
For the {\GCC} compiler the Xcode~\cite{xcode} package has to be installed.
To install {\CMake}, external portation tools such as
Fink~\cite{fink}, DarwinPorts~\cite{darwinports}, 
or MacPorts~\cite{macports} have to be used. 

The build process of {\ViennaMath} is similar to Linux.

\subsection{Windows}
In the following the procedure is outlined for \texttt{Visual Studio}: Assuming
that an {\OpenCL} SDK and {\CMake} are already installed, Visual Studio solution
and project files can be created using {\CMake}:
\begin{itemize}
\item Open the {\CMake} GUI.
\item Set the {\ViennaMath} base directory as source directory.
\item Set the \texttt{build/} directory as build directory.
\item Click on 'Configure' and select the appropriate generator
(e.g.~\texttt{Visual Studio 9 2008}).
\item Click on 'Configure' again.
\item Click on 'Generate' in order to let {\CMake} generate the project files for you.
\item The project files can now be found in the {\ViennaMath} build directory,
where they can be opened and compiled with Visual Studio (provided that the
include and library paths are set correctly, see
Sec.~\ref{sec:viennamath-installation}).
\end{itemize}

























