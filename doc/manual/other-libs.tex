\chapter{Interfaces to Other Libraries} \label{chap:other-libs}
{\ViennaCL} aims at compatibility with as many other libraries as possible.
This is on the one hand achieved by using generic implementations of the individual algorithms,
and on the other hand by providing the necessary wrappers.

The interfaces to third-party libraries provided with {\ViennaCL} are explained in the following subsections.
Please feel free to suggest additional libraries for which an interface should be shipped with {\ViennaCL}.

Since it is unlikely that all third-party libraries for which {\ViennaCL} provides interfaces are installed
on the target machine, the wrappers are disabled by default. To selectively enable the wrappers,
the appropriate preprocessor constants \lstinline|VIENNACL_WITH_XXXX| have to be defined \emph{prior to any
\lstinline|\#include| statements for {\ViennaCL} headers}. This can for example be assured by passing the
preprocessor constant directly when launching the compiler. With \lstinline|GCC| this is for instance
achieved by the \lstinline|-D| switch.

\section{Boost.uBLAS}
Since all types in {\ViennaCL} have the same interface as their counterparts in {\ublas},
most code written for {\ViennaCL} objects remains valid when using {\ublas} objects.
\begin{lstlisting}
//Option 1: Using ViennaCL:
using namespace viennacl;
using namespace viennacl::linalg;

//Option 2: Using ublas:
//using namespace boost::numeric::ublas;

matrix<float>            dense_matrix(5,5);
vector<float>            dense_vector(5,5);
compressed_matrix<float> sparse_matrix(1000, 1000);

//fill with data:
dense_matrix(0,0) = 2.0;
....

//run solvers
vector<float> result1 = solve(dense_matrix, dense_vector, upper_tag());
vector<float> result2 = viennacl::linalg::solve(sparse_matrix,
                                                dense_vector, cg_tag());
\end{lstlisting}
The above code is valid for either the {\ViennaCL} namespace declarations, or the {\ublas} namespace. Note that the
iterative solvers are not part of {\ublas} and therefore the explicit namespace specification is required. More examples
for the exchangability of {\ublas} and {\ViennaCL} can be found in the tutorials in the \texttt{examples/tutorials/} folder.

When using the iterative solvers, the preprocessor constant \texttt{VIENNACL\_WITH\_UBLAS} must be defined prior to any other {\ViennaCL} include statements.
This is essential for enabling the respective wrappers.

\TIP{Refer in particular to \texttt{iterative-ublas.cpp} for a complete example on iterative solvers using {\ublas} types.}

\section{\Eigen}
To copy data from {\Eigen} \cite{eigen} objects (version 3.x.y) to {\ViennaCL}, the \texttt{copy()}-functions are used just as for {\ublas} and STL types:
\begin{lstlisting}
 //from Eigen to ViennaCL
 viennacl::copy(eigen_vector,       vcl_vector);
 viennacl::copy(eigen_densematrix,  vcl_densematrix);
 viennacl::copy(eigen_sparsematrix, vcl_sparsematrix);
\end{lstlisting}
In addition, the STL-compliant iterator-version of \texttt{viennacl::copy()} taking three arguments can be used for copying vector data.
Here, all types prefixed with \texttt{eigen} are {\Eigen} types, the prefix \texttt{vcl} indicates {\ViennaCL} objects.
Similarly, the transfer from {\ViennaCL} back to {\Eigen} is accomplished by
\begin{lstlisting}
 //from ViennaCL to Eigen
 viennacl::copy(vcl_vector,       eigen_vector);
 viennacl::copy(vcl_densematrix,  eigen_densematrix);
 viennacl::copy(vcl_sparsematrix, eigen_sparsematrix);
\end{lstlisting}

The iterative solvers in {\ViennaCL} can also be used directly with {\Eigen} objects:
\begin{lstlisting}
  using namespace viennacl::linalg; //for brevity of the following lines
  eigen_result = solve(eigen_matrix, eigen_rhs, cg_tag());
  eigen_result = solve(eigen_matrix, eigen_rhs, bicgstab_tag());
  eigen_result = solve(eigen_matrix, eigen_rhs, gmres_tag());
\end{lstlisting}
When using the iterative solvers with {\Eigen}, the preprocessor constant \texttt{VIENNACL\_WITH\_EIGEN} must be defined prior to any other {\ViennaCL} include statements.
This is essential for enabling the respective wrappers.

\TIP{Refer to \texttt{iterative-eigen.cpp} and \texttt{eigen-with-viennacl.cpp} for complete examples.}



\section{MTL 4}
The following lines demonstate how {\ViennaCL} types are filled with data from {\MTL} \cite{mtl4} objects:
\begin{lstlisting}
 //from Eigen to ViennaCL
 viennacl::copy(mtl4_vector,       vcl_vector);
 viennacl::copy(mtl4_densematrix,  vcl_densematrix);
 viennacl::copy(mtl4_sparsematrix, vcl_sparsematrix);
\end{lstlisting}
In addition, the STL-compliant iterator-version of \texttt{viennacl::copy()} taking three arguments can be used for copying vector data.
Here, all types prefixed with \texttt{mtl4} are {\MTL} types, the prefix \texttt{vcl} indicates {\ViennaCL} objects.
Similarly, the transfer from {\ViennaCL} back to {\MTL} is accomplished by
\begin{lstlisting}
 //from ViennaCL to MTL4
 viennacl::copy(vcl_vector,       mtl4_vector);
 viennacl::copy(vcl_densematrix,  mtl4_densematrix);
 viennacl::copy(vcl_sparsematrix, mtl4_sparsematrix);
\end{lstlisting}

Even though {\MTL} provides its own set of iterative solvers, the iterative solvers in {\ViennaCL} can also be used:
\begin{lstlisting}
  using namespace viennacl::linalg; //for brevity of the following lines
  mtl4_result = solve(mtl4_matrix, mtl4_rhs, cg_tag());
  mtl4_result = solve(mtl4_matrix, mtl4_rhs, bicgstab_tag());
  mtl4_result = solve(mtl4_matrix, mtl4_rhs, gmres_tag());
\end{lstlisting}
Our internal tests have shown that the execution time of {\MTL} solvers is equal to {\ViennaCL} solvers when using {\MTL} types.

When using the iterative solvers with {\MTL}, the preprocessor constant \texttt{VIENNACL\_WITH\_MTL4} must be defined prior to any other {\ViennaCL} include statements.
This is essential for enabling the respective wrappers.

\TIP{Refer to \texttt{iterative-mtl4.cpp} and \texttt{mtl4-with-viennacl.cpp} for complete examples.}
