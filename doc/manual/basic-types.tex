
\chapter{Basic Types} \label{chap:basics}

Since C++ is a statically typed language \cite{Stroustrup:cpp}, the basic mathematical building blocks such as constants or variables are represented as types.
The possiblity of manipulations at compiletime or runtime is accomplished by essentially
two different implementations of these primitives.
Basic types for runtime evaluations are discussed first, since their interface and handling is potentially more familiar to average C++ programmers.
Sec.~\ref{sec:compiletime-types} then provides an overview of the basic types used for compiletime manipulations.

The main include file for {\ViennaMath} is \lstinline|viennamath/expression.hpp| and includes all the types discussed in the remainder of this chapter.

\TIP{Include \lstinline|viennamath/expression.hpp| to make all {\ViennaMath} types available.}

\TIP{Note that all types reside in namespace \lstinline|viennamath|. The namespace is not written explicitly in the following, thus either \lstinline|viennamath::| prefixes or certain \lstinline|using| declarations need to be added by the user in order to make the code valid.}


\section{Types Evaluated at Runtime} \label{sec:runtime-types}
Common to all types represented at runtime is that they inherit from the same abstract base class 
and can thus be accessed and manipulated using a pointer to that interface.
The interface is not fixed a-priori and can be adjusted via a template parameter, which is in the following called \lstinline|InterfaceType|.
Library users should use the expression wrapper objects discussed next, because it provides an automatic memory management and does not involve complicated pointer manipulation.

%%%%%%%%%%%%%%%%%%%%%%%%%%%%%%%%%%%%%%


 \subsection{Expression Wrapper \lstinline|expr|}
The main expression wrapper type in {\ViennaMath} is \lstinline|rt_expr<InterfaceType>|.
The prefix \lstinline|rt| refers to \emph{runtime} and aids in distinguishing between types processed at runtime, and types processed at compiletime.
In most cases, the default parameter for the runtime interface \lstinline|InterfaceType| is used, in which case users would have to write
\begin{lstlisting}
 rt_expr<> my_expression = /* any expression here */;
\end{lstlisting}
for instantiating an expression wrapper object \lstinline|my_expression|.
In order to avoid users from having to write the \lstinline|rt_| and the lower-than and greater-than signs,
there is a convenience shortcut \lstinline|expr| provided. The previous code line thus becomes
\begin{lstlisting}
 expr my_expression = /* any expression here */;
\end{lstlisting}
The \lstinline|expr|-type can be evaluated and manipulated using operator overloads.
For example, the addition of two expressions is accomplished by
\begin{lstlisting}
 expr ex1 = /* any expression here */;
 expr ex2 = /* any expression here */;
 expr result = ex1 + ex2;
\end{lstlisting}
The initalization of expression objects is accomplished by any of the fundamental types discussed in the next subsections.
Note that objects of type \lstinline|expr| are default-constructible, yet they can only be used after an expression has been assigned to them.

%%%%%%%%%%%%%%%%%%%%%%%%%%%%%%%%%%%%%%

 \subsection{Constant}
Constants in C++ have their own types \lstinline|double|, \lstinline|long|, etc.
These types can be used with {\ViennaMath} directly. 
In order to also represent constants using a pointer to the runtime interface, a separate class \lstinline|rt_constant<NumericT, InterfaceType>| is provided.
The template parameter \lstinline|NumericT| denotes the underlying numerical type such as \lstinline|double|, \lstinline|long|, or high precision types.
There is again a convenience shortcut \lstinline|constant| provided for the case of the commonly used \lstinline|rt_constant<double>|, hence a user can write code such as
\begin{lstlisting}
 constant pi = 3.1415;
 constant pi_squared = pi * pi;
\end{lstlisting}
An exemplary use with the expression wrapper \lstinline|expr| is
\begin{lstlisting}
 constant pi = 3.1415;
 expr pi_squared = pi * pi;
 expr result = pi + pi_squared;
\end{lstlisting}

%%%%%%%%%%%%%%%%%%%%%%%%%%%%%%%%%%%%%%

 \subsection{Variable}
A mathematical variable in {\ViennaMath} is modeled by \lstinline|rt_variable<InterfaceType>|.
and refers to the mapping
\begin{align*}
 \left( x_0, x_1, \ldots, x_{N-1}  \right) \mapsto x_j \ ,
\end{align*}
where the value of $j$ is provided to the constructor of the variable.
By default, the index $j=0$ is used. Any vector type offering access to its values using \lstinline|operator[]| such as \lstinline|std::vector<T>| can be used for an evaluation of the variable or a compounded expression.

A simple example leading to the mapping $(x,y) \mapsto x(y+\pi)$ using the types introduced so far is as follows:
\begin{lstlisting}
 constant pi = 3.1415;
 variable x(0);
 variable y(1);
 expr f = x * (y + pi);
\end{lstlisting}
An evaluation of \lstinline|f| at $(1,2)$ can be accomplished by using the functor interface provided by an overload of the parenthesis operator
and the {\ViennaMath} helper function \lstinline|make_vector()|, which conveniently creates a suitable vector for evaluation.
\begin{lstlisting}
 std::cout << f( make_vector(1,2) ) << std::endl;  //prints 5.1415
\end{lstlisting}


%%%%%%%%%%%%%%%%%%%%%%%%%%%%%%%%%%%%%%

 \subsection{Unary Expression}
Mappings of the form $x \mapsto \sin(x)$ are modeled by the \lstinline|rt_unary_expr<InterfaceType>| class.
Thus, they represent a unary function acting on a constant, a variable or an expression. An overview of the unary functions provided with {\ViennaMath} is given in Tab.~\ref{tab:unary-functions}.

\begin{table}
\centering
\begin{tabular}{|l|l||l|l|}
\hline
Name & {\ViennaMath} Function   & Name & {\ViennaMath} Function \\
\hline
Exponential   & \lstinline|exp()| & Modulus & \lstinline|fabs()| \\
Sine          & \lstinline|sin()| & Square Root & \lstinline|sqrt()| \\
Cosine        & \lstinline|cos()| & Natural Logarithm & \lstinline|log()| \\
Tangent       & \lstinline|tan()| & Logarithm, Base 10 & \lstinline|log10()| \\
\hline
\end{tabular}
\caption{Overview of unary functions defined in {\ViennaMath}.\label{tab:unary-functions}}
\end{table}

\TIP{Function names in Tab.~\ref{tab:unary-functions} are intentionally chosen such that they coincide with the standard functions for floating point types.
When calling these functions with floating point types, compilation might fail due to ambiguity. In such case the namespace should be specified explicitly.}

Typically, unary expressions are not instantiated explicitly by the library user.
Instead, they are generated implicitly by one of the unary functions and then assigned to an object of type \lstinline|expr| as in the following example:
\begin{lstlisting}
 variable x;
 expr g = sin(2.0 * x);  // wraps a unary expression into 'g'
\end{lstlisting}

 \subsection{Binary Expression}
Similar to unary expressions, binary expressions at runtime are mostly handled in the background only.
They are created whenever one of the operator overloads for addition, subtraction, multiplication, or division is triggered.
In particular, the argument \lstinline|2.0 * x| to \lstinline|sin()| in
\begin{lstlisting}
 expr g = sin(2.0 * x);
\end{lstlisting}
is a binary expression. Binary expressions are central for compile time evaluations in Sec.~\ref{sec:compiletime-types}.

 \subsection{Expression Vector}
For the cases where a vector-valued expression is required, a user can either instantiate a vector of \lstinline|expr|, which allows for storing multiple scalar-valued function only,
or use the \lstinline|rt_vector_expr<InterfaceType>| class provided by {\ViennaMath}. A convenience shortcut \lstinline|vector_expr| is provided.
The benefit of using the \lstinline|vector_expr| class is that it provides the usual operator overloads directly:
\begin{lstlisting}
 variable x(0), y(1);
 vector_expr vec(3); vec[0] = x; vec[1] = y; vec[2] = x + y;
 vector_expr vec2 = x * vec + y * vec;
\end{lstlisting}
The dot-product of two vector-valued expressions is provided as well:
\begin{lstlisting}
 expr h = vec * vec2;
\end{lstlisting}



\section{Types Evaluated at Compiletime} \label{sec:compiletime-types}
The runtime types discussed in the previous section enable a convenient handling of expressions.
However, there are numerous runtime dispatches required when evaluating such runtime expressions, which are too costly in a high performance setting.
The compiletime types discussed in this section avoid any additional runtime dispatches and their use thus result in faster code in general.
This gain in performance comes at the price of a few additional restrictions:
Since the expression is entirely encoded in the type, there is no equivalent to \lstinline|expr| in order to assign an expression to a another object\footnote{The new C++11 standard addresses this issue and provides the \lstinline|auto| keyword for automatic type deduction. However, {\ViennaMath} intentionally does not use any C++11 features yet.}.
Furthermore, compilation times increase due to the additional work to be done for the compiler.
Excessive use of compiletime evaluations and manipulations can even result in minutes to hours of compilation time, even though this is rarely encountered in practice.
Another complication stems from the fact that no floating point template arguments are allowed, thus reducing any compiletime calculations to integer calculations.
Fractional numbers can be emulated this way, but they cannot resolve all problems.

 \subsection{Constant}
Since no floating point type is allowed as template argument, only integer values \lstinline|val| are represented by the class \lstinline|ct_constant<val>|.
Operators are overloaded in the same way as for the runtime evaluation types in Sec.~\ref{sec:runtime-types}. One example of a compiletime calculation is given as follows:
\begin{lstlisting}
 ct_constant<2> c2;   //the constant '2'
 ct_constant<5> c5;   //the constant '5'
 std::cout << c2 + c5 << std::endl;  //prints '7' (computed at compiletime)
\end{lstlisting}
Note that \lstinline|ct_constant<>| can in principle also be mixed with ordinary constants such as
\begin{lstlisting}
 std::cout << 2 + c5 << std::endl;  //prints '7'
\end{lstlisting}
However, depending on the optimization capabilities of the C++ compiler used, ordinary constants may or may not be used for compiletime computations, while the compiler is forced to do it in the introductory snippet.

\TIP{A general guideline is to use \lstinline|ct_constant<val>| for encoding an integer \lstinline|val| already known at compile time rather than writing the value explicitly in code. }

 \subsection{Variable}
A mathematical variable for compiletime manipulations is represented by \lstinline|ct_variable<id>|, where \lstinline|id| refers to the coordinate entry in the evaluation vector.
The meaning of \lstinline|id| is identical to the constructor argument of a \lstinline|variable| in the runtime case. 

Operators are again overloaded as usual. For example, consider
\begin{lstlisting}
 ct_variable<0> x;
 ct_variable<1> y;
 std::cout << x * y << std::endl;
\end{lstlisting}


 \subsection{Unary Expression}
The unary functions in Tab.~\ref{tab:unary-functions} can also be called with compiletime types.
The corresponding type for the compiletime representation is provided by \lstinline|ct_unary_expr<E, OP>|, where \lstinline|E| refers to the expression on which the unary function encoded by the tag \lstinline|OP| acts.
Unary operation tags start with \lstinline|op_| and are defined in \lstinline|viennamath/compiletime/unary_op_tags.hpp|. Their type name can be deduced from the function names in Tab.~\ref{tab:unary-functions} by adding the prefix.
Note that all unary functions are evaluated at runtime, because the underlying C-functions are called for evaluation. 
For example, the type \lstinline|T| of the compiletime unary expression
\begin{lstlisting}
 ct_variable<0> x;
 T t = sin(x);
\end{lstlisting}
is \lstinline|ct_unary_expr< ct_variable<0>, op_sin<NumericT> >|, where \lstinline|NumericT| is the floating point type used for the evaluation at runtime (typically \lstinline|double|).



 \subsection{Binary Expression}
The binary expression \lstinline|ct_binary_expr<L, OP, R>| with left hand side expression \lstinline|L|, operation tag \lstinline|OP| and right hand side expression \lstinline|R| are the main types for building more complex expressions.
Currently, four binary operations are supported: addition (with tag \lstinline|op_plus<NumericT>|), subtraction (\lstinline|op_minus<NumericT>|), multiplication (\lstinline|op_mult<NumericT>|), and division (\lstinline|op_div<NumericT>|).
Similar to unary expressions, binary expressions are seldomly set up by hand.
Two examples of binary expressions are as follows:
\begin{lstlisting}
 ct_binary_expr< ct_variable<0>,    // x
                 op_plus<double>,   // +
                 ct_variable<1> >   // y  

 ct_binary_expr< ct_constant<1>,    // 1
                 op_div<double>,    // /
                 ct_variable<0> >   // x
\end{lstlisting}

Typical uses of binary expressions are within the manipulation of compiletime expressions in metafunctions. As an example, outputting the first term of a polynomial is considered:
\begin{lstlisting}
 ct_variable<0> x;
 ct_variable<1> y;
 print_first( x*y + x*x*y - y*y );
\end{lstlisting}
Only two versions of the \lstinline|print_first| function are required.
The first one recursively traverses the binary expression along the left hand side argument:
\begin{lstlisting}
 template <typename L, typename OP, typename R>
 print_first(ct_binary_expr<L, OP, R> const & b)
 { print_first(b.lhs()); } //recursion along left hand side
\end{lstlisting}
The recursion terminates with a general implementation for printing the left-most entry:
\begin{lstlisting}
 template <typename T>
 print_first(T const & t)
 { std::cout << t << std::endl; }
\end{lstlisting}

\NOTE{If a binary operation consists of one object for compiletime and one for runtime evaluation, the compiletime object is converted to a runtime object and then processed as usual in the runtime setting.}



 %\subsection{Expression List}



