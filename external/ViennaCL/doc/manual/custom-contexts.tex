\chapter{Using ViennaCL in User-Provided OpenCL Contexts} \label{chap:custom-contexts}

Many projects need similar basic linear algebra operations, but essentially operate in their own {\OpenCL} context.
To provide the functionality and convenience of {\ViennaCL} to such existing projects,
existing contexts can be passed to {\ViennaCL} and memory objects can be wrapped into the basic linear algebra types \lstinline|vector|, \lstinline|matrix| and \lstinline|compressed_matrix|.
This chapter is devoted to the description of the necessary steps to use {\ViennaCL} on contexts provided by the library user.

\TIP{An example of providing a custom context to {\ViennaCL} can be found in \texttt{examples/tutorial/custom-contexts.cpp}}

\section{Passing Contexts to ViennaCL}
{\ViennaCLversion} is able to handle an arbitrary number of contexts, which are identified by a key value of type \lstinline|long|.
By default, {\ViennaCL} operates on the context identified by $0$, unless the user switches the context, cf.~Chapter \ref{chap:multi-devices}.

According to the {\OpenCL} standard, a context contains devices and queues for each device. Thus, it is assumed in the following that
the user has successfully created a context with one or more devices and one or more queues per device.

In the case that the context contains only one device \lstinline|my_device| and one queue \lstinline|my_queue|, the context can be passed to {\ViennaCL} with the code
\begin{lstlisting}
 cl_context my_context = ...;     //a context
 cl_device_id my_device = ...;    //a device in my_context
 cl_command_queue my_queue = ...; //a queue for my_device

 //supply existing context 'my_context'
 // with one device and one queue to ViennaCL using id '0':
 viennacl::ocl::setup_context(0, my_context, my_device, my_queue);
\end{lstlisting}
If a context ID other than \texttt{0}, say, \lstinline|id| is used, the user-defined context has to be selected using
\begin{lstlisting}
 viennacl::ocl::switch_context(id);
\end{lstlisting}

It is also possible to provide a context with several devices and multiple queues per device. To do so, the device IDs have to be stored in a STL vector
and the queues in a STL map:
\begin{lstlisting}
 cl_context my_context = ...;   //a context

 cl_device_id my_device1 = ...; //a device in my_context
 cl_device_id my_device2 = ...; //another device in my_context
 ...

 cl_command_queue my_queue1 = ...; //a queue for my_device1
 cl_command_queue my_queue2 = ...; //another queue for my_device1
 cl_command_queue my_queue3 = ...; //a queue for my_device2
 ...

 //setup existing devices for ViennaCL:
 std::vector<cl_device_id> my_devices;
 my_devices.push_back(my_device1);
 my_devices.push_back(my_device2);
 ...

 //setup existing queues for ViennaCL:
 std::map<cl_device_id,
          std::vector<cl_command_queue> > my_queues;
 my_queues[my_device1].push_back(my_queue1);
 my_queues[my_device1].push_back(my_queue2);
 my_queues[my_device2].push_back(my_queue3);
 ...

 //supply existing context with multiple devices
 //and queues to ViennaCL using id '0':
 viennacl::ocl::setup_context(0, my_context, my_devices, my_queues);
\end{lstlisting}
It is not necessary to pass all devices and queues created within a particular context to {\ViennaCL}, only those which {\ViennaCL} should use have to be passed.
{\ViennaCL} will by default use the first queue on each device. The user has to care for appropriate synchronization between different queues.

\TIP{{\ViennaCL} does not destroy the provided context automatically upon exit. The user should thus call \lstinline|clReleaseContext()| as usual for destroying the context.}

\section{Wrapping Existing Memory with ViennaCL Types}
Now as the user provided context is supplied to {\ViennaCL}, user-created memory objects have to be wrapped into {\ViennaCL} data-types in order to use the full functionality.
Typically, one of the types \lstinline|scalar|, \lstinline|vector|, \lstinline|matrix| and \lstinline|compressed_matrix| are used:
\begin{lstlisting}
 cl_mem my_memory1 = ...;
 cl_mem my_memory2 = ...;
 cl_mem my_memory3 = ...;
 cl_mem my_memory4 = ...;
 cl_mem my_memory5 = ...;

 //wrap my_memory1 into a vector of size 10
 viennacl::vector<float> my_vec(my_memory1, 10);

 //wrap my_memory2 into a row-major matrix of size 10x10
 viennacl::matrix<float> my_matrix(my_memory2, 10, 10);

 //wrap my_memory3 into a CSR sparse matrix with 10 rows and 20 nonzeros
 viennacl::compressed_matrix<float> my_sparse(my_memory3,
                                              my_memory4,
                                              my_memory5, 10, 10, 20);

 //use my_vec, my_matrix, my_sparse as usual
\end{lstlisting}
The following has to be emphasized:
\begin{itemize}
 \item Resize operations on {\ViennaCL} data types typically results in the object owning a new piece of memory.
 \item copy() operations from CPU RAM usually allocate new memory, so wrapped memory is ``forgotten''
 \item On construction of the {\ViennaCL} object, \lstinline|clRetainMem()| is called once for the provided memory handle. Similarly, \lstinline|clReleaseMem()| is called as soon as the memory is not used any longer.
\end{itemize}

\NOTE{The user has to ensure that the provided memory is larger or equal to the size of the wrapped object.}

\NOTE{Be aware the wrapping the same memory object into several different {\ViennaCL} objects can have unwanted side-effects.
In particular, wrapping the same memory in two {\ViennaCL} vectors implies that if the entries of one of the vectors is modified, this is also the case for the second.}
