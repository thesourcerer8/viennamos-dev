
\chapter{Additional Features}
Various features of {\ViennaMath}, which are not necessarily standard features of a symbolic math library, are covered in this chapter.
Additional feature requests should be sent to
\begin{center}
\texttt{viennamath-support$@$lists.sourceforge.net} 
\end{center}

  \section{\LaTeX{} Output}
Since {\ViennaMath} encourages a high-level description and manipulation of the underlying mathematical problem formulation in source code, it is natural to
generate \LaTeX{} code from {\ViennaMath} expressions for debugging purposes. The generated code can be copy\&paste'd to LaTeX rendering webpages or used for
the automatic generation of program log files in the form of a \LaTeX{} document.

All conversion is carried out by a separate converter object of type \lstinline|rt_latex_translator<InterfaceType>| as defined in
\lstinline|viennamath/manipulation/latex.hpp|. A convenience shortcut \lstinline|latex_translator| is available for the default runtime expression interface.
Conversion is triggered by providing the expression to be converted to the functor:
\begin{lstlisting}
 latex_translator  to_latex;

 expr f = sqrt( x + y );
 to_latex( f );     //returns the string '\sqrt{x_{0}+x_{1}}'
\end{lstlisting}
By default, variables are printed as $x_0$, $x_1$, etc. This and other output routines can be customized by using the \lstinline|customize()| member function
of the converter. For example, to print 'x' and 'y' instead of 'x\_\{0\}' and 'x\_\{1\}', the code
\begin{lstlisting}
 to_latex.customize(x, "x");
 to_latex.customize(y, "y");
\end{lstlisting}
is sufficient. Similar customizations can be applied for the output of types and features described in the remainder of this chapter.

\NOTE{The \LaTeX{} generator works with runtime expression types only. Thus, compiletime expression types need to be converted to runtime expression types
first.}

  \section{Differential Operators}
For enabling dimension-independent programming, dimension-independent mathematical differential operations are also provided with {\ViennaMath}.
In {\ViennaMathversion}, the gradient and the divergence operators are provided by the functions \lstinline|grad()| and \lstinline|div()| respectively:
\begin{lstlisting}
 expr u = grad(x+y);
 expr v = grad(x-y);
 expr w = div(grad(x*x + y*y));
\end{lstlisting}
Note that expression containing differential operators cannot be evaluated directly, since a coordinate system needs to be specified first.

A coordinate system is applied to the previous expressions by using the free function \lstinline|apply_coordinate_system()|, which is defined in the header file \lstinline|viennamath/manipulation/apply_coordinate_system.hpp|. The first function argument is a tag identifying the coordinate system (either of type \lstinline|cartesian<1>|, \lstinline|cartesian<2>|, or \lstinline|cartesian<3>| for a Cartesian coordinate system in one, two or three dimensions). The second function argument is the expression to which the coordinate system should be applied:
\begin{lstlisting}
 apply_coordinate_system(cartesian<1>(), u); //returns 1 + y
 apply_coordinate_system(cartesian<2>(), v); //returns (1, -1)
 apply_coordinate_system(cartesian<3>(), w); //returns 4
\end{lstlisting}
Differential operators are very handy in combination with function symbols explained in Sec.~\ref{sec:function-symbols}.

%%%%%%%%%%%%%%%
  \section{Integration Symbols}
In certain cases the form of an integral expression is known, but the actual integration domain is determined at some later stage. 
Here, a symbolic integration domain of type \lstinline|symbolic_interval| identified by an ID can be specified and substituted by the final integration interval later on.
For example, the integral $\int_\Omega x^2 \: \mathrm{d} \Omega$, where $\Omega$ is specified at some later point, is specified using {\ViennaMath} as
\begin{lstlisting}
 expr my_integral = integral(symbolic_interval(), x*x);
\end{lstlisting}
The interface is again such that it can be used with both runtime and compiletime types. For simplicity, the resulting integral expression is here assigned to a runtime expression \lstinline|my_integral|. An ID can be provided to the constructor of \lstinline|symbolic_interval| for distinguishing between different symbolic intervals. By default, an ID of $0$ is used.

In order to substitute the symbolic interval with the actual integration interval, the function \lstinline|substitute()| as explained in Sec.~\ref{sec:substitute} is used.
Since the replacement consists of both the integration interval and the variable over which integration is to be carried out, the replacement argument is packed into a \lstinline|pair| as defined in the C++ STL. Thus, in order to specify $\Omega$ as the interval $[0,1]$ with an integration over the $x$-variable, the code
\begin{lstlisting}
 expr my_integral2 = substitute( symbolic_interval(),
                                 std::make_pair(interval(0, 1), x),
                                 my_integral);
\end{lstlisting}
is sufficient. Note that in the current release of {\ViennaMath} only the substitution with a one-dimensional integration domain is supported, but no nested integrals are possible yet.

\TIP{Note that \lstinline|std::make_pair()| is defined in the header \lstinline|<utility>|.}


%%%%%%%%%%%%%%%
  \section{Function Symbols} \label{sec:function-symbols}
For discretization schemes based on weak formulations of partial differential equations it is appropriate to work with abstract functions rather than with concrete expressions.
For example, the weak form of the Poisson equation,
\begin{align} \label{eq:weak-poisson}
 \int_\Omega \nabla u \cdot \nabla v \: \mathrm{d} x = \int_\Omega fv \: \mathrm{d} x 
\end{align}
for all test functions in a certain test space $\mathcal{V}$, is formulated for functions $u$ and $v$, which are during the discretization replaced by certain trial and test functions, which finally yields a system of linear equations. 
Such function labels (\emph{function symbols}) are modeled in {\ViennaMath} by the type \lstinline|rt_function_symbol<InterfaceType>| at runtime (with convenience shortcut \lstinline|function_symbol| for the default runtime interface type) and by \lstinline|ct_function_symbol<T>| at compiletime, where \lstinline|T| is a tag identifying the function symbol.

As a simple example, the expression $uv$ is considered, where $u$ is then substituted with the expression $(1+x)$ and $v$ is replaced by $(1-x)$:
\begin{lstlisting}
 function_symbol u(0);
 function_symbol v(1);
 expr f = u * v;
 expr g = substitute(u, 1.0 + x,
                     substitute(v, 1.0 - x, f)
                    );    // g becomes (1+x)*(1-x)
\end{lstlisting}
The constructor arguments denote the function symbol ID used for distinguishing the individual function symbols.

Reconsidering the weak form \eqref{eq:weak-poisson}, function symbols at compiletime can use any arbitrary tag class for identification.
{\ViennaMath} provides the predefined tags \lstinline|unknown_tag<id>| and \lstinline|test_tag<id>|, where the integer template parameter \lstinline|tag| is used for distinguishing between several function symbols of the same tag. The previous code snippet rewritten for compiletime manipulation thus becomes
\begin{lstlisting}
 ct_function_symbol< unknown_tag<0> > u;
 ct_function_symbol<    test_tag<0> > v;
 substitute(u, 1.0 + x,
            substitute(v, 1.0 - x, u*v)
           );    // returns (1+x)*(1-x)
\end{lstlisting}
Note that the result type of \lstinline|substitute()| encodes the result \lstinline|(1+x)*(1-x)|, thus the result is usually directly passed to another function (e.g.~\lstinline|eval()|) in order to avoid writing the return type explicitly.

As a final example, the weak form \eqref{eq:weak-poisson} (with $f \equiv 1$ for simplicity) is specified directly as a {\ViennaMath} runtime expression and converted to {\LaTeX}-code using the functionality presented in this chapter:
\begin{lstlisting}
 function_symbol u(0);
 function_symbol v(1);
 equation weak_form = make_equation( integral( symbolic_interval(),
                                               grad(u) * grad(v) ),
                                     integral( symbolic_interval(), v ) );
 latex_translator to_latex;
 std::cout << to_latex(weak_form) << std::endl;
\end{lstlisting}


\TIP{ Have a look at \texttt{ViennaFEM}~(\texttt{http://viennafem.sourceforge.net/}) if you are interested in a software package using {\ViennaMath} for the finite element method. }



