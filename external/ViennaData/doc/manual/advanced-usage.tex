\chapter{Advanced Usage} \label{chap:advanced-usage}

\NOTE{This documentation hasn't been updated to the interface of the upcoming ViennaData 1.1.0 release yet}

While Chapter \ref{chap:basic-usage} outlines the basic usage of {\ViennaData}, this chapter focuses
on customizing the library to the user's demands and the particular environment. It is explained how
the object identification can be changed, how the internal storage of data can be customized and data
accesses can be tuned for maximum speed.

\section{Retrieving IDs from Objects} \label{sec:id-retrieval}
In order to associate data with objects, one has to distinguish between different objects. The only portable identification mechanism
is the address of an object, thus this is the default identification mechanism {\ViennaData}.
However, object addresses can be randomly scattered over the address space, which leads to tree-based data structures in order to
provide a reasonably fast lookup of data.

However, it can be the case that there are additional means for identification provided.
Let us consider a class \lstinline|MyClass| equipped with a member function \lstinline|id()|, which
is used to identify objects. The following code configures {\ViennaData} to use this member function
as identification mechanism for all objects
\begin{lstlisting}
namespace viennadata
{
 namespace config
 {
  template <>
  struct object_identifier<MyClass>
  {
   typedef object_provided_id     tag;
   typedef long                   id_type;

   static id_type get(MyClass const & obj) { return obj.id(); }
  };
 }
}
\end{lstlisting}
Summing up, a specialization for the template class \lstinline|object_identifier| has to be provided
for each class, containing the following:
\begin{itemize}
 \item A type definition for a type \lstinline|tag|. This line is identical for all user-provided overloads.
 \item A type definition for a type \lstinline|id_type|. This specifies the type of the ID.
 \item A static member function \lstinline|get|, which returns the ID from an object.
\end{itemize}
The specialization can be placed at any point in the source code.

\section{Dense Data} \label{sec:dense-data}
By default, {\ViennaData} internally uses a tree-based layout similar to
\begin{lstlisting}
 std::map<ObjType *, DataContainer>
\end{lstlisting}
to store and retrieve data for objects of type \lstinline|MyClass| by the use of a data container type \lstinline|DataContainer|.
This induces lookup-costs of the order $\log N$, where $N$ denotes the number of objects for which data is stored.

However, if the range of IDs of objects of a type \lstinline|MyClass| is known to be in the range $[0, \ldots, N]$ for some reasonable value of $N$,
a tree-based layout can be configured to be of type
\begin{lstlisting}
 std::vector<DataContainer>
\end{lstlisting}
This requires that a custom identification mechanism is set up as outlined in the previous section. Moreover, the
custom \lstinline|id_type| must be compatible with being used as an array index. Since such a storage scheme is only memory efficient,
if the number of objects is comparable to $N$, we refer to this scheme as \emph{dense data storage}.
To change the default tree-based data storage scheme to the dense storage scheme for objects of type \lstinline|MyClass|, the lines
\begin{lstlisting}
namespace viennadata
{
 namespace config
 {
  template <typename KeyType, typename DataType>
  struct storage<KeyType, DataType, MyClass>
  {
   typedef dense_data_tag    tag;
  };
 }
}
\end{lstlisting}
are sufficient. Due to the template specialization, a fine-grained control over which storage to use for the various types of data and keys is readily provided. Thus, it is possible to use both dense and sparse storage schemes simultaneously for associating data with a particular object.

For convenience, a number of macros is provided, which simplify the overloading process:
\begin{lstlisting}
 VIENNADATA_ENABLE_DENSE_DATA_STORAGE_FOR_KEY_DATA_OBJECT(...)
 VIENNADATA_DISABLE_DENSE_DATA_STORAGE_FOR_KEY_DATA_OBJECT(...)

 VIENNADATA_ENABLE_DENSE_DATA_STORAGE_FOR_KEY_DATA(...)
 VIENNADATA_DISABLE_DENSE_DATA_STORAGE_FOR_KEY_DATA(...)
 VIENNADATA_ENABLE_DENSE_DATA_STORAGE_FOR_KEY_OBJECT(...)
 VIENNADATA_DISABLE_DENSE_DATA_STORAGE_FOR_KEY_OBJECT(...)
 VIENNADATA_ENABLE_DENSE_DATA_STORAGE_FOR_DATA_OBJECT(...)
 VIENNADATA_DISABLE_DENSE_DATA_STORAGE_FOR_DATA_OBJECT(...)

 VIENNADATA_ENABLE_DENSE_DATA_STORAGE_FOR_KEY(...)
 VIENNADATA_ENABLE_DENSE_DATA_STORAGE_FOR_DATA(...)
 VIENNADATA_ENABLE_DENSE_DATA_STORAGE_FOR_OBJECT(...)
\end{lstlisting}
The parameter to be passed to the macros are the key type, the data type and/or the object type as indicated by the macro name tail.
Note that the macros have to be placed in global namespace and the template overloading rules apply.

Similar to tree-based schemes, dense data storage also offers automatic memory management. However, to reduce reallocations,
the user may reserve the required memory manually.
To reserve data memory for data of type \lstinline|DataType| with keys of type \lstinline|KeyType| for up to $42$ objects of the same type as \lstinline|obj|,
one writes
\begin{lstlisting}
  viennadata::reserve<KeyType,DataType>(42)(obj);
\end{lstlisting}
Similarly, memory for other data, key and object types is reserved.


\section{Type-Based Key Dispatch} \label{sec:compiletime-keys}
As already shown in Chapter \ref{chap:basic-usage}, {\ViennaData} can be configured to perform a purely type-based key dispatch for data access.
The impact on performance can be considerable, especially when compared to potentially costly string comparisons.
To enable a compile time dispatch for a key of type \lstinline|SomeKeyType|, the lines
\begin{lstlisting}
namespace viennadata
{
 namespace config
 {
  template <>
  struct key_dispatch<SomeKeyType>
  {
    typedef type_key_dispatch_tag    tag;
  };
 }
}
\end{lstlisting}
are sufficient. One may then omit the key parameter for the API functions provided by {\ViennaData}, cf.~Chapter \ref{chap:basic-usage}.

A convenience macro is again provided:
\begin{lstlisting}
 VIENNADATA_ENABLE_TYPE_BASED_KEY_DISPATCH(...)
\end{lstlisting}
where the parameter is the the key type. Note that the macro has to be placed in global namespace.


\TIP{When using a compile time key dispatch based on types, the key type is not required to fulfill the LessThan-Comparable concept.}

\section{Default Data Types for Keys} \label{sec:default-valuetype}
For the cases where only one data type per key type is used, {\ViennaData} can be provided with a default data type for each key type.
This is especially common, when using a type-based key dispatch for data access. Consider for example
\begin{lstlisting}
 viennadata::access<front_color, RGBColor>()(obj1) = RGBColor(255,42,0);
 viennadata::access<back_color,  RGBColor>()(obj1) = RGBColor(0,42,255);
 viennadata::access<edge_color,  RGBColor>()(obj1) = RGBColor(255,0,42);
\end{lstlisting}
Since the key classes are already named appropriately and always the some data type \lstinline|RGBColor| is used,
setting \lstinline|RGBColor| as default data type for the respective keys allows to make the code more compact.
This can be enabled along the lines
\begin{lstlisting}
namespace viennadata
{
 namespace config
 {
  template <>
  struct default_data_for_key<front_color>
  {
    typedef RGBColor    type;
  };
  // and similarly for back_color, edge_color
 }
}
\end{lstlisting}
Now, the initial code snippet can be written more compactly as
\begin{lstlisting}
 viennadata::access<front_color>()(obj1) = RGBColor(255,42,0);
 viennadata::access< back_color>()(obj1) = RGBColor(0,42,255);
 viennadata::access< edge_color>()(obj1) = RGBColor(255,0,42);
\end{lstlisting}
If no default data type is specified, but the second argument is nevertheless omitted, a compile time error is thrown.

A convenience macro for the default data type is provided by
\begin{lstlisting}
 VIENNADATA_ENABLE_DEFAULT_DATA_TYPE_FOR_KEY(...)
\end{lstlisting}
where the two parameters are the key type and the data type. Note that the macro has to be placed in global namespace.
